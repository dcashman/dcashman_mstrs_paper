Secure random number generation is an important problem which has received a generous amount of attention.  As mentioned, changes to the Linux kernel were made shortly after Heninger et al published their discovery of the boot-time entropy hole on which our research is based \cite{pnqs}.  K Mowery et al. also followed that research with a paper proposing three novel methods of generating large amounts of entropy very early on in the boot process even for headless and embedded devices\cite{kmowery}.  Other evaluations of entropy provision following Heniger et al.'s paper are Hennebert et al. \cite{2013_entropy} and Herrewege et al \cite{vansecure}.  Our work also benefited greatly from Gutterman et al., who have been very active in this space by doing the first analysis of the Linux PRNG \cite{linuxPRNG}, exposing weak randomness in session keys \cite{java_sessid}, and even evaluating the Windows PRNG \cite{windows}.  We have also been fortunate to have previous work looking into shortcomings of past PRNG implementations such as the analysis of the netscape browser SSL issues in \cite{netscape} and follow ups such as \cite{yilek}.

Our investigation has also opened further avenues of inquiry.  Our discovery of the implicit entropy generated by different request ordering could be measured across different systems to determine just how much entropy it conveys.  We also only touched upon a very small subset of security-mechanisms dependent on randomness.  Initial investigation into wpa-supplicant, for example, suggests that it uses /dev/urandom to seed its own PRNG which is based on the Linux PRNG.  Finally, an analysis of these attacks and entropy measurements should be conducted on devices \textit{in the wild} to determine their general feasibility and impact.
